\section{Snippets}

\subsection{Snippets}

\begin{frame}{Basic frame}{Subtitle}
    \dots
\end{frame}
\note{
Add some notes\dots
}

\begin{frame}{Citations and references}{cite, label and ref commands}
    Eq. (\ref{eq:bellman}) define the Bellman equation \cite{bellman1956dynamic}
    \begin{equation}
        V(x) = \max_{a \in \Gamma (x) } \{ F(x,a) + \beta V(T(x,a)) \}  \label{eq:bellman}
    \end{equation}
\end{frame}
\note{
}


\begin{frame}{Lists}{itemize, enumerate and description commands}
    \begin{itemize}
        \item item 1
        \item item 2
        \item \dots
    \end{itemize}

    \begin{enumerate}
        \item item 1
        \item item 2
        \item \dots
    \end{enumerate}

    \begin{description}
        \item[First] item 1
        \item[Second] item 2
        \item[Last] \dots
    \end{description}
\end{frame}
\note{
}


\begin{frame}{Colors}{color environment}
    \begin{small}
    small
    \end{small}

    \begin{footnotesize}
    footnotesize
    \end{footnotesize}

    \begin{scriptsize}
    scriptsize
    \end{scriptsize}

    \begin{tiny}
    tiny
    \end{tiny}
\end{frame}
\note{
}


\begin{frame}{Fonts color}{color environment}
    {\color{red} Red}
    {\color{green} Green}
    {\color{blue} Blue}
\end{frame}
\note{
}


\begin{frame}{Centered image}{includegraphics command}
    \begin{center}
        \includegraphics[width=.80\linewidth]{fig/test}
    \end{center}
\end{frame}
\note{
}

\begin{frame}{Subfigures}{figure, subfigure and includegraphics commands}
    \begin{figure}
        \centering
        \subfigure{\includegraphics[width=.30\linewidth,height=.20\linewidth]{fig/test}}~
        \subfigure{\includegraphics[width=.30\linewidth,height=.20\linewidth]{fig/test}}~
        \subfigure{\includegraphics[width=.30\linewidth,height=.20\linewidth]{fig/test}}
        ~\\
        \subfigure{\includegraphics[width=.30\linewidth,height=.20\linewidth]{fig/test}}~
        \subfigure{\includegraphics[width=.30\linewidth,height=.20\linewidth]{fig/test}}~
        \subfigure{\includegraphics[width=.30\linewidth,height=.20\linewidth]{fig/test}}
    \end{figure}
\end{frame}
\note{
}


\begin{frame}{Blocks}{block command}
    \begin{block}{Block 1}
        Blablabla
    \end{block}

    ~\\

    \begin{block}{Block 2}
        Blablabla
    \end{block}
\end{frame}
\note{
}


\begin{frame}{Equations}
    $$
        V(x) = \max_{a \in \Gamma (x) } \{ F(x,a) + \beta V(T(x,a)) \}  \label{eq:bellman}
    $$

    \[
        V(x) = \max_{a \in \Gamma (x) } \{ F(x,a) + \beta V(T(x,a)) \}  \label{eq:bellman}
    \]

    \begin{equation}
        V(x) = \max_{a \in \Gamma (x) } \{ F(x,a) + \beta V(T(x,a)) \}  \label{eq:bellman}
    \end{equation}
\end{frame}
\note{
}



\begin{frame}{Equation array}{eqnarray command}
    \begin{tiny}
        \begin{eqnarray*}
            \mbox{Expectation of N} & = & \sum_{i=1}^{n} \E(Z_i) \\
                                    & = & \sum_{i=1}^{n} \frac{\gamma}{d^{\beta/2}} \frac{ c(d)^\beta }{i^{\alpha\beta}} \\
                                    & = & \frac{\gamma}{d^{\beta/2}} c(d)^\beta \sum_{i=1}^{n} \frac{1}{i^{\alpha\beta}} \\
                                    & = & z \\
                                    & ~ & \\
        \end{eqnarray*}
        \begin{eqnarray}
            \mbox{Variance of N} & = & \sum_{i=1}^{n} V(Z_i) \\
                                 & \leq & \sum_{i=1}^{n} \E(Z_i) ~~~~~~~ (\mbox{as } V(Z_i) \leq \E(Z_i)) \\
                                 & \leq & z  \nonumber
        \end{eqnarray}
    \end{tiny}
\end{frame}
\note{
}


\begin{frame}{Matrices}
    \begin{small}
        $$
            A_{m,n} =
            \begin{pmatrix}
                a_{1,1} & a_{1,2} & \cdots & a_{1,n} \\
                a_{2,1} & a_{2,2} & \cdots & a_{2,n} \\
                \vdots  & \vdots  & \ddots & \vdots  \\
                a_{m,1} & a_{m,2} & \cdots & a_{m,n}
            \end{pmatrix}
        $$

        $$
            M =
            \begin{bmatrix}
                \frac{5}{6} & \frac{1}{6} & 0           \\[0.3em]
                \frac{5}{6} & 0           & \frac{1}{6} \\[0.3em]
                0           & \frac{5}{6} & \frac{1}{6}
            \end{bmatrix}
        $$

        $$
            M = \bordermatrix{~ & x & y \cr
                              A & 1 & 0 \cr
                              B & 0 & 1 \cr}
        $$
    \end{small}
\end{frame}
\note{
}


\begin{frame}{Systems of equation array}
    \[
        f(n) = \left\{
        \begin{array}{l l}
            n/2      & \quad \text{if $n$ is even}\\
            -(n+1)/2 & \quad \text{if $n$ is odd}
        \end{array} \right.
    \]
\end{frame}
\note{
}


\begin{frame}{Mathematical programming}{with align}
    \begin{align}
        \max        & \quad z = 4 x_1 + 7 x_2    \notag \\
        \text{s.t.} & \quad 3 x_1 + 5 x_2 \leq 6 \label{constraint1}\\
                    & \quad   x_1 + 2 x_2 \leq 8 \label{constraint2}\\
                    & \quad   x_1, x_2 \geq 0    \notag
    \end{align}
\end{frame}
\note{
}


\begin{frame}{Mathematical programming}{with alignat}

    % see http://tex.stackexchange.com/questions/75108/how-to-edit-the-linear-programming-in-latex
    % and http://tex.stackexchange.com/questions/83918/labeled-linear-program-with-labeled-equations-and-wide-objective-function

    % see http://latex.wikia.com/wiki/List_of_LaTeX_environments for explanations about alignat

    % The alignat environment can be used to align equations, and explicitly
    % specify the number of "equation" columns. An equation column has two
    % parts, separated by the equals-sign. Essentially, this is an array with
    % alternating right-aligned and left-aligned columns. The required
    % parameter of alignat is the maximum number of ampersands in a row plus 1,
    % and then divided by 2. One use of alignat is to explicitly specify the
    % amount of horizontal space between columns by including the required
    % spacing in the first row.

    % \rlap is used since the operators in the first equation have no business
    % being aligned with the rest.

    \begin{scriptsize}
        %\begin{alignat*}{6}   % argument = at least (the number of '&' + 1) / 2
        %    \text{Max}  \quad \rlap{$z = x_1 + 12x_2$}                                       \\               
        %    \text{s.t.} \quad & 13 & x_1 & {}+{} &  & x_2 & {}+{} & 12 & x_3 & ~ \leq ~ & 5  \\
        %                      &    & x_1 &       &  &     & {}+{} &    & x_3 & ~ \leq ~ & 16 \\
        %                      & 15 & x_1 & {}+{} &  & x_2 &       &    &     & ~ =    ~ & 14 \\
        %                      & \rlap{$x_j \geqslant 0,\; j=1,2,3.$}
        %\end{alignat*}

        \begin{alignat*}{5}  % argument = at least (the number of '&' + 1) / 2
            \text{Max}  \quad \rlap{$z = x_1 + 12x_2$}                                   \\               
            \text{s.t.} \quad & & 13 x_1 & {}+{} &  x_2 & {}+{} & 12 x_3 & ~ \leq ~ & 5  \\
                              & &    x_1 &       &      & {}+{} &    x_3 & ~ \leq ~ & 16 \\
                              & & 15 x_1 & {}+{} &  x_2 &       &        & ~ =    ~ & 14 \\
                              & \rlap{$x_j \geq 0, ~ j=1,2,3.$}
        \end{alignat*}

        \begin{alignat}{5}   % argument = at least (the number of '&' + 1) / 2
            \text{Max}  \quad \rlap{$z = x_1 + 12x_2$}                                  \nonumber  \\
            \text{s.t.} \quad & & 13 x_1 & {}+{} & x_2 & {}+{} & 12 x_3 & ~ \leq ~ & 5  \label{c1} \\
                              & &    x_1 &       &     & {}+{} &    x_3 & ~ \leq ~ & 16 \label{c2} \\
                              & & 15 x_1 & {}+{} & x_2 &       &        & ~ =    ~ & 14 \label{c3} \\
                              & \rlap{$x_j \geq 0, ~ j=1,2,3.$}                         \nonumber
        \end{alignat}
    \end{scriptsize}

\end{frame}
\note{
}


\begin{frame}{Animations}
    \only<1| handout:0> {
        Slide 1
        \begin{itemize}
            \item \dots
            \item \dots
        \end{itemize}
    }

    \only<2| handout:0> {
        Slide 2
        \begin{itemize}
            \item \dots
            \item \dots
        \end{itemize}
    }

    \only<3| handout:0> {
        Slide 3
        \begin{itemize}
            \item \dots
            \item \dots
        \end{itemize}
    }
\end{frame}
\note{
}

% Algorithmic commands:
%  \STATE <text>
%  \IF{<condition>} \STATE{<text>} \ENDIF
%  \FOR{<condition>} \STATE{<text>} \ENDFOR
%  \FOR{<condition> \TO <condition> } \STATE{<text>} \ENDFOR
%  \FORALL{<condition>} \STATE{<text>} \ENDFOR
%  \WHILE{<condition>} \STATE{<text>} \ENDWHILE
%  \REPEAT \STATE{<text>} \UNTIL{<condition>}
%  \LOOP \STATE{<text>} \ENDLOOP
%  \REQUIRE <text>
%  \ENSURE <text>
%  \RETURN <text>
%  \PRINT <text>
%  \COMMENT{<text>}
%  \AND, \OR, \XOR, \NOT, \TO, \TRUE, \FALSE
\begin{frame}{Algorithms}{algorithmic command}
    \begin{scriptsize}
        \begin{algorithmic}
            \REQUIRE ~\\
                     $\langle \S, \A, T, R \rangle$, an MDP\\
                     $\discount$, the discount factor\\
                     $\epsilon$, the maximum error allowed in the utility of any state in an iteration
            %\ENSURE ~\\
            %        $U, U'$, vector of utilities for states in $\S$, initially zero\\
            %        $\delta$, the maximum change in the utility of any state in an iteration\\
            \STATE \hspace{-1em}\textbf{Local variables:}\\
                    $U, U'$, vector of utilities for states in $\S$, initially zero\\
                    $\delta$, the maximum change in the utility of any state in an iteration\\
                    ~

            \REPEAT
                \STATE $U \leftarrow U'$
                \STATE $\delta \leftarrow 0$
                \FORALL{$\s \in \S$}
                    \STATE $U'[\s] \leftarrow R[\s] + \discount \max_a \sum_{\s'} T(s,a,s') U[s']$
                    \IF{$|U'[\s] - U[\s]| > \delta$}
                        \STATE $\delta \leftarrow |U'[\s] - U[\s]|$
                    \ENDIF
                \ENDFOR
            \UNTIL{$\delta < \epsilon(1-\discount)/\discount$}
            ~

            \RETURN $U$
        \end{algorithmic}
    \end{scriptsize}
\end{frame}
\note{
}


% ATTENTION: pour insérer un paragraphe en verbatim dans une frame, il est nécessaire d'ajouter l'option "fragile" à la frame.
% http://mcclinews.free.fr/latex/introbeamer/cadre.html
% http://blog.guiling.fr/index.php?post/2009/12/15/Utiliser-verbatim-dans-un-document-Beamer
\begin{frame}[fragile]
    \frametitle{Verbatim}  % required to be here and not above...
    To insert a verbatim paragraph, the frame have to be declared "fragile".
    The title has to be written in frametitle command, not as argument of frame (I don't know why\dots).

\begin{verbatim}
    .--.
   |o_o |
   |:_/ |
  //   \ \
 (|     | )
/'\_   _/`\
\___)=(___/

# gcc -o hello hello.c
\end{verbatim}

\end{frame}
\note{
}


\begin{frame}[allowframebreaks]  % for long listings
    \frametitle{Listings}  % as for verbatim, frametitle is required to be here and not as argument above...
    \begin{small}
        \lstinputlisting[language=Python]{listings/test.py}
        %\lstinputlisting[language=Python, firstline=2, lastline=5]{listings/test.py}
    \end{small}
\end{frame}
\note{
}


\begin{frame}{Table}{tabular command}
    \begin{small}
        \begin{tabular}{|l|c|c|}
        \hline
                                  & $\gamma=1$ (small noise)                      & $\gamma<1$ (large noise) \\
        \hline
        Proved rate for R-EDA     & $\frac{1}{\beta} \leq {\color{blue} \alpha}$  & $\frac{1}{2\beta} \leq {\color{blue} \alpha}$ \\
        \hline
        Former lower bounds       & ${\color{blue} \alpha} \leq 1$                & ${\color{blue} \alpha} \leq 1$ \\
        \hline
        R-EDA experimental rates  & ${\color{blue} \alpha} = \frac{1}{\beta}$     & ${\color{blue} \alpha} = \frac{1}{2\beta}$ \\
        \hline
        \hline
        Rate by active learning   & ${\color{blue} \alpha} = \frac{1}{2}$        & ${\color{blue} \alpha} = \frac{1}{2}$   \\
        \hline
        \end{tabular}
    \end{small}
\end{frame}
\note{
}


\begin{frame}{Multi-columns}{columns and column commands}
    \begin{columns}
        \begin{column}{0.5\textwidth}

            Blablabla

        \end{column}
        \begin{column}{0.5\textwidth}

            \begin{center}
                \includegraphics[width=.99\linewidth]{fig/test}
            \end{center}

        \end{column}
    \end{columns}
\end{frame}
\note{
}


\begin{frame}{URL}
    \url{http://www.inria.fr/}
\end{frame}
\note{
}
